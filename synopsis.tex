\documentclass[12pt,a4paper]{article}
\usepackage[margin=1in,left=1.5in,top=1.5in,includefoot]{geometry}
\usepackage{lipsum}% http://ctan.org/pkg/lipsum
\usepackage{xcolor}% http://ctan.org/pkg/xcolor
\usepackage{fancyhdr}% http://ctan.org/pkg/fancyhdr


\newcommand{\mycomment}[1]{}

%global definitions
\definecolor{headerRed}{HTML}{823909}


\fancypagestyle{myPageStyle}{%
	\fancyhf{}% Clear all headers/footers
	\fancyhead[L]{Social Platform for Auctioning Tour Plans with Interested Tourists to Agencies}% Header Centred
	\fancyfoot[C]{-\thepage-}% Footer Centred
	\renewcommand{\headrulewidth}{3pt}% 2pt header rule
	\renewcommand{\headrule}{
		\hbox to\headwidth{
		\color{headerRed}\leaders\hrule height \headrulewidth\hfill}}
	\renewcommand{\footrulewidth}{0pt}
	
	\fancyfoot{}
	\fancyfoot[L]{Dept. of CSE, AJIET, Mangalore}
	\fancyfoot[R]{\thepage}
	\renewcommand{\footrulewidth}{3pt}
	\renewcommand{\footrule}{\hbox to\headwidth{\color{headerRed}\leaders\hrule height \footrulewidth\hfill}}
	% No footer rule
}
\setlength{\headheight}{21pt}%


\begin{document}


	\tableofcontents
	\pagestyle{myPageStyle}
	\newpage

	
	
	
	\section*{CHAPTER 1}
	\subsection*{\protect \centering INTRODUCTION}
	\addcontentsline{toc}{section}{\numberline{}INTRODUCTION}
	\pagenumbering{arabic}
	\pagestyle{fancy}
	\pagestyle{myPageStyle}

	\hspace{0.5cm}The main idea behind the project is to develop an android application which will help tourists to find the better place at one instant. The long-time which Tourists waste on searching for the better places like hotels, museums, parks etc for their enjoyment in the new city which is totally
	unknown to them will get reduced, if they use this application.\\
	
	
	Hence this idea was very new and useful for all those who love to travel in a new city on a
	regular basis. The project is about tourist guide system how the tourist will get best use of the
	application according to his/her point of interest. Developing a Location Based Tourist Guide
	Application is basically a GPS enhanced travel expo application, which allows the users to
	participate in a self-guided tour of a specific area. It will also display detailed information about
	specific features linked to their current position.\\

	\hspace{0.5cm} Next Generation Location based services for mobile devices is a mobile computing application
	that provides information functionality to users based on their geographical location. In
	addition to showing the nearby restaurants type of application, it contains some extra features
	such as pro-actively push only relevant information to users to help speed up decisions and
	activities, encourage sharing of location-based information such as photos and reviews
	generated by other service providers and users.
	
	\newpage
	
	\section*{CHAPTER 2}
	\subsection*{\protect \centering BRIEF LITERATURE SURVEY}
	\addcontentsline{toc}{section}{\numberline{}BRIEF LITERATURE SURVEY}
	\pagenumbering{arabic}
	\pagestyle{fancy}
	\pagestyle{myPageStyle}
	
	\begin{itemize}
		\item The main goal of this study is to explore factors which influence city destination choice
		among young people in Serbia.
		\item In order to achieve this, we conducted a survey consisting of 20 different items
		influencing the choice of city destination.
		\item The results indicate four motivating factors extracted by factor analysis, from which
		good hospitality and restaurant service seems to be the major motivating factor.
		\item The results also show that respondents belonging to the age group of under 25 give
		more importance to Information and promotion as well as to good hospitality and
		restaurant service than those belonging to older age groups.
		\item According to the World Tourism Organization [UNWTO], the tourism sector covers
		travel services related to recreational, leisure or business purposes.
		\item During the past decades, the tourism sector has experienced continuous growth and
		diversification, becoming one of the most dynamic sectors of the global economy.
		\item Nowadays, tourism represents worldwide a significant driver for economic growth,
		accounting for about 10% of the global economic activity.
		\item Nowadays, tourism represents worldwide a significant driver for economic growth,
		accounting for about 10% of the global economic activity.
		\item Tourism can serve as a tool towards increasing government revenues through taxes and
		improving residents’ quality of life in a destination. There are, however, potential
		negative environmental and social impacts that must be managed for tourism to fulfil
		its promises.
		\item Nearly everyone goes on a vacation and a Tourism management system would play a
		vital role in planning the perfect trip.
		\item There are multiple positive effects derived from tourism such as the creation of jobs, its
		capability to fix the population to the territory, or its ability to diversify agricultural
		production in certain areas.
		\item The tourism management system allows the user of the system access all the details
		such as weather, location, events, etc.
		\item The main purpose is to help tourism companies to manage customer and hotels etc.
		\item The system can also be used for both professional and business trips.
		\item The proposed system maintains centralized repository to make necessary travel
		arrangements and to retrieve information easily.
		
	\end{itemize}
	\newpage

\section*{CHAPTER 3}
\subsection*{\protect \centering PROBLEM FORMULATION}
\addcontentsline{toc}{section}{\numberline{}PROBLEM FORMULATION}
\pagenumbering{arabic}
\pagestyle{fancy}
\pagestyle{myPageStyle}

\subsubsection*{3.1 PROBLEM FORMULATION}
	\begin{itemize}
	\item The aim of this project is to suggest tour places for users.
	\item It is difficult to decide a tour plan for users to travel to their dream destination.
	\item On an average 1 billion people travel throughout the world. So it's difficult for each one
	of them to plan their tour. To solve this issue this app helps users to have a proper tour
	plan.
	\item Lot of time is consumed while planning for the tour. Cost planning could also be
	increased than expected
	\end{itemize}


\subsubsection*{3.3 METHODOLOGY}
TourPeak is the name of this application, The work flow of TourPeak is as follows:
\begin{itemize}
	\item Firstly, users as well as tourist agencies will have a login/register screen shown to
	them.
	\item Based on the users credentials they will be logged on to the respective portal i.e.
	either agency portal or tourists portal or admin portal.
	\item The following are the 3 different users and features available to them:
	
	\begin{enumerate}
	\item Tourists
		\begin{enumerate}
	    	\item Can browse all proposed tour plans.
			\item Can like and share tour plans.
			\item Can create custom tour plans i.e. roadmap of their plan.
		\end{enumerate}
	\item Tourist Agencies
		\begin{enumerate}
		\item Has all features of tourists.
		\item Can bid for proposing tour plans to the tourists that have liked
		the plans.
		\end{enumerate}
	\item Admin
	\begin{enumerate}
		\item Regulates the tourists as well as agencies.
		\item Makes sure bidding goes through in a fair way.
		\item Handles all the maintenance and other updates to the application.
	\end{enumerate}

	\end{enumerate}
	\item As a tourist, you will be shown the most trending tour first, the interface looks similar
to Instagram reels or YouTube shorts.
\item Based on location of tourist, the plans in his feed will be adjusted such that the plans
starting at the tourist’s current location will be shown.
\item Tourists, will also have a create button available to them using which they’ll be
directed to the roadmap portal wherein they can create their own custom tour plans for
sharing their ideas and plans of what their dream tour is.
\item Tourists Agencies will be shown the most trending tours and they can bid on
proposing plans for the tourists interested in said plan.
\item Agency that wins the bid is given the right to propose plans to all the interested
tourists.
\end{itemize}

\newpage

\section*{CHAPTER 4}
\subsection*{\protect \centering HARDWARE AND SOFTWARE REQUIRMENT}

	\subsubsection*{ 4.1 HARDWARE REQUIRMENTS}
		\begin{itemize}
			\item RAM: 1GB or above
			\item Hard disk: 100MB or above
			\item Processor: 2.4 GHZ or above 
		\end{itemize}
		\subsubsection*{ 4.2 SOFTWARE REQUIREMENTS}
	\begin{enumerate}
		\item Flutter
		\item MySQL
		\item NodeJS
		\item Android Studio
		\item React Native
	\end{enumerate}
			\subsubsection*{4.3 SOFTWARE INTERFACES}
		\begin{enumerate}
		\item Language: JavaScript, Dart
		\item Editor: Visual Studio Code 
	\end{enumerate}

\newpage
\section*{CHAPTER 5}
\subsection*{\protect \centering APPLICATIONS}
\begin{itemize}
	\item Showcasing personal tour plans, user can plot their own ideas and personalise
	other’s plans. Users can also share other’s plans and promote them in other platforms.
	\item Optimizing plan as per user satisfaction, if the user is satisfied with plan proposed
	by other user’s, he can either go with the designed plan, else the user can propose his
	own ideas on a plan that he wants to go on.
	\item Sharing of data to tourist agencies to provide said plans, the tour agencies would
	contact the user based on their (user) liking plans and negotiate plan prices for the
	tour. The agents bid on plans interesting the user and the user can choose which
	agency is providing budget-friendly prices.
	\item Social Media for tour plans, acts as a bridge between user and tour agencies, where
	interested agencies can bid their price on certain plan that the user wants to go with.
	User can like the plans posted by other users, and share in other platform for better
	recognition of plans
	\item Cost effective tour plans, the plans designed by the users are usually cost effective
	and budget-friendly for people who prefer pocket-friendly tour plans, and
	accommodation provided by the travel agencies.
\end{itemize}

\newpage
\section*{CHAPTER 6 }
\subsection*{\protect \centering ADVANTAGES AND DISADVANTAGES}
\subsubsection*{6.1 ADVANTAGES}
\begin{itemize}
\item Allows users to share their interests and take a chance at realizing their dream
destinations at nominal costs.
\item Trending Tour plans will be displayed allowing tourist agencies to dish out between
each other for giving customers the said tour at decreasing costs.
\item Allows users to showcase their interested destinations.
\item Tourist agencies can customize their plans based on interests shown.
\item Users are given a portal to request for tours as needed.
\item Users can get access to usual tours as well along with a chance to having their dream
tour realized in the way the want it.
\end{itemize}

\subsubsection*{6.2 DISADVANTAGES}
\begin{itemize}
		\item The user that suggests the initial tour plan needs to put in some effort to create it.
		\item More competition between tourist agencies as all have access to the portal.
		\item Tourist Agencies need to have dedicated people assigned for the same, to efficiently
	use this portal.
		\item Some learning is involved to use the app for customers as well as tourists
\end{itemize}
\newpage
\subsection*{\protect \centering REFERENCES}
\begin{itemize}
	\item Travel Package Recommendation System: A Literature Review, Himani M. Mishra1, Dr . Ms. V. M. Deshmukh
	
	\item  Smart Tourism Recommendation Model: A Systematic Literature, Review Choirul
	Huda, Arief Ramadhan*, Agung Trisetyarso, Edi Abdurachman, Yaya Heryadi.
	\item Mobile Application Development in the Tourism Industry and its Impact on On-Site
	Travel Industry and its Impact on On-Site Travel Behavior, Moritz Christian.
	\item A Survey on Tourist Trip Planning Systems, Kadri Sylejmani and Agni Dika,
	University of Prishtina, Kosova.
	\item The Preferences of Potential Tourists in Utilizing Travel Agencies and Travel
	Application.
	\item  Online Travel Purchasing: A Literature Review, Article in Journal of Travel and
	Tourism Marketing · November 2013.
	\item A systematic literature review for the tourist trip design problem: Extensions, solution
	techniques and future research lines Extensions, solution techniques and future
	research lines Jos Ruiz-Meza.
	\item Mobile Apps in Tourism Communication: The Strengths and Weaknesses on Tourism
	Trips.
	\item  Designing Apps for Tourists: A Case Study by Caio Cristo, Maria Gabriela and Lucas
	Santos.
\end{itemize}


\end{document}
